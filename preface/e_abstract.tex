% !Mode:: "TeX:UTF-8"

% 英文摘要
\begin{Abstract}
Under the background of radar signal classification, this paper focuses on sea/land clutter recognition for Over-the-Horizon Radar~(OTHR)~and specific emitter identification using deep learning methodology, which are as follows.

(1)~The results of OTHR sea/land clutter recognition can be used for coordinate registration, which can help improve the accuracy of target localization.
Based on the analysis of spectrum data, a six-layer convolution neural network classifier is constructed.
The original spectrum data is preprocessed including cleansing, trimming, and fusion.
Adaptive moment estimation method is used to train the network, which can give a stable learning rate.
The classification threshold is adjusted according to the features of clutter. The results are evaluated by the classification accuracy and the matching accuracy with actual map.
Experimental results show that the clutter recognition algorithm based on deep convolution neural network has high recognition accuracy even in a poor ionospheric environment, from $96\%-99\%$.

(2)~In order to avoid labeling large amounts of data,
a deep embedding convolution method for sea/land clutter clustering is proposed.
Spatial locations and time sequences information is added into original clutter spectrum data.
The network consists of convolutional layer, deconvolutional layer and fully connected layer.
A convolutional autoencoders structure is developed to learn embedded features in an end-to-end way.
The experimental results show the clustering accuracy of deep embedding convolution clustering method can reach $97.85\%$.

(3)~In order to recognize emitters and identify unknown types of emitters in complex electromagnetic environment, a specific emitter identification method based on deep learning is proposed.
By analyzing the signals of the specific emitters and using slice of ambiguity function as input feature vectors,
a 10-layer one-dimensional convolution neural network is constructed as the primary classifier to classify the input signals firstly. Unknown types of emitters are identified by a support vector machine classifier as Meta-Recognition.
Experiments of meteorological radars from aircrafts show that the classification accuracy of the proposed algorithm is high, the recognition rate of the known categories and unknown categories(the number of unknown categories is more than 3) reaches $97\%$ and $93\%$, respectively.

\begin{Keywords}
Radar Signal Classification, Sea/Land Clutter Recognition for Over-The-Horizon Radar, Specific Emitter Identification, Deep Learning, Convolutional Neural Networks
\end{Keywords}
\end{Abstract}
