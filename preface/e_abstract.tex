% 英文摘要
\begin{Abstract}
Under the background of radar signal classification, this paper focuses on sea/land clutter recognition for Over-the-Horizon Radar(OTHR) and specific emitter identification using deep learning methodology, as follows:


1)The results of OTHR sea/land clutter recognition can be used for coordinate registration, which can help improve the target positioning accuracy effectively.
Based on the analysis of sea/land clutter spectrum data, a six-layer deep convolution neural network classifier is constructed. The original spectrum data is preprocessed by cleansing, trimming, and fusion.
Adaptive Moment Estimation method is used to train the network, which can give a stable learning rate.
As a real-world project, the classification threshold is adjusted according to the features of clutter. The results are evaluated by the classification accuracy and the matching accuracy with actual map.
Experimental results show that the clutter recognition algorithm based on deep convolution neural network still has high recognition accuracy under poor ionospheric environment.

2)In order to overcome the high cost of labeling large amounts of data and the difficult to label some data  accurately manually, this paper further proposed a deep embeddeding convolution method for sea/land clutter clustering.
Spatial locations and time sequences information is added to original clutter spectrum data. The network contains convolutional layer, deconvolutional layer and fully connected layer.
A convolutional autoencoders structure is developed to learn embedded features in an end-to-end way.
The experimental results show the effectiveness of deep embeddeding convolution clustering method.

3)In order to solve the problems of recognizing emitters in similar radar signal parameters and the existence of unknown types emitterin complex electromagnetic environment, this paper proposed a specific emitter identification method based on deep learning.
By analyzing the signal of the specific emitters and using the ambiguity function slice as feature vectors,
a 10-layer one-dimensional convolution neural network is constructed as the primary classifier to classify the input signals firstly. Unknown types of specific emitters are identified by a support vector machine classifier as Meta-Recognition.
The classification accuracy of the algorithm proposed in this paper is high in the experiments of meteorological radars from multiple aircraft.



	\begin{Keywords}
		Radar Signal Classification, Sea/Land Clutter Recognition for Over-The-Horizon Radar, Specific Emitter Identification, Deep Learning, Convolutional Neural Networks
	\end{Keywords}
\end{Abstract}