% 中文摘要
\begin{abstract}
	% 中文 \lipsum[2-3]

	% 摘要——学位论文的简短陈述,应说明该研究工作的目的意义、研究方法、研究成果等,其重点是研究成果,摘要字数一般为硕士论文700~1000字
本文以雷达信号分类识别为主要应用背景,利用深度学习的思想与方法,重点研究了天波超视距雷达地海杂波识别和雷达辐射源识别这两个实际工程问题,具体工作如下:

1)利用天波超视距雷达地海杂波识别结果可以实时准确获得天波雷达的坐标配准修正系数,进而有效提高目标定位精度。
本文在对地海杂波频谱实际数据深入分析的基础上,对原始频谱数据首先进行清洗、裁剪、融合等预处理步骤,然后构建了一个具有六层的深度卷积神经网络分类器并利用具有自适应学习率的算法进行学习,同时结合地海杂波这一实际工程问题对分类阈值进行调整,最后对输出结果以分类精度和与实际地图的匹配精度两个层面进行衡量。
实验结果表明,基于深度卷积神经网络的地海杂波识别算法在电离层环境较差的情况下仍具有较高的识别精度。

2)为了克服大量数据进行标记的成本过高以及部分数据难以人工准确标注的问题,本文进一步提出了基于深度嵌入卷积的地海杂波聚类方法。
通过在地海杂波数据中包含空间、时间信息,添加卷积层、卷积转置层和全连接层,设计了一个以端到端方式训练的卷积自编码器来从未标记的数据中学习特征,并将KL散度与重构损失两部分结合设计了一个新的损失函数。
利用天波超视距雷达的实际地海杂波数据的实验结果表明了深度嵌入卷积聚类算法的有效性。

3)针对复杂电磁环境下辐射源的识别面临的电磁信号干扰大、雷达信号参数相近以及工程中存在未知类别的辐射源等问题与挑战,本文提出了基于深度学习的辐射源个体识别算法。
通过对现有辐射源信号进行分析,利用其模糊函数切片这一脉内细微特征作为训练样本,构建了一个具有10层的一维卷积神经网络作为主分类器来对输入信号做初次分类,其后跟随一个支持向量机分类器作为Meta-Recognition,实现对辐射源中未知类别的辨别。
通过多架飞机的气象雷达辐射源的实际数据证明了本文算法分类的准确性。



	\begin{keywords}
		雷达信号分类, 天波雷达地海杂波识别, 雷达辐射源分类, 深度学习, 卷积神经网络
	\end{keywords}
\end{abstract}