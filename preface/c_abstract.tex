% 中文摘要
\begin{abstract}
	% 中文 \lipsum[2-3]

	% 摘要——学位论文的简短陈述,应说明该研究工作的目的意义、研究方法、研究成果等,其重点是研究成果,摘要字数一般为硕士论文700~1000字
本文以雷达信号分类识别为主要应用背景,重点研究了天波超视距雷达地海杂波识别和雷达辐射源识别与未知分类辨别这两个问题,具体工作如下:

1)基于深度学习的天波超视距雷达地海杂波识别。提出了基于卷积神经网络天波雷达地海杂波识别的新算法,该方法主要克服了传统的阈值识别方法或支持向量机算法根据经验从频谱数据中提取特征,导致操作复杂度高,分类精度低的缺点。同时,我们将我们的算法与传统算法和支持向量机算法进行了对比,实验结果表明,我们的方法在地海杂波识别问题上更加有效以及抗干扰性能更强,同时发现利用不同时间的同一区域的频谱数据进行融合可以很大程度上提高分类精度。在更高精度的识别结果的帮助下,我们可以得到更加精确的修正系数,可以为目标定位问题提供非常大的帮助。

2)基于深度学习的辐射源识别。针对复杂电磁环境下辐射源的识别面临的电磁信号干扰大、雷达信号参数相近等问题与挑战,通过对现有辐射源信号进行分析,利用其脉内细微特征作为训练样本,通过大量的样本,智能地判断各特征的权重,通过赋予不同的权重在保留雷达个体特征的情况下,使得识别准确率有了较大的进步,并设计了meta识别器,实现了对辐射源中未知分类的辨别。

3)基于深度嵌入卷积聚类方法的地海杂波无监督分类。结合卷积神经网络与自编码器提出了深度嵌入卷积聚类方法,设计了以端到端方式训练的卷积自编码器来从未标记的数据中学习特征,通过利用卷积自编码器进行提取并在损失函数中将KL散度与重构损失两部分结合,有效提高了聚类的精度。并利用天波超视距雷达的实际地海杂波数据进行了实验,结果验证了深度嵌入聚类算法的有效性。


	\begin{keywords}
		深度学习, 天波超视距雷达, 地海杂波识别, 辐射源分类, 未知分类, 卷积神经网络
	\end{keywords}
\end{abstract}