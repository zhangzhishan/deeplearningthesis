% !Mode:: "TeX:UTF-8"

时光荏苒,光阴似箭。两年多的研究生生涯即将结束,我的校园生活也即将画上句号。研究生阶段的生活和本科时候,发生了很大的变化,实验室给提供了更加优异的环境,让我可以学习到更多的知识。

首先,我要感谢我的导师潘泉老师。虽然和潘老师接触的时间有限,但是每次和潘老师交流都给我很大的帮助。潘老师宽广的视野总是能一针见血的发现问题所在,让我深受启发。潘老师对学术的热情和认真的态度一直鼓励着我用最严谨的态度进行学习和工作,他渊博的学识、开阔的思维以及循循善诱的教学方法让我深受裨益。

在硕士期间,王增福老师和兰华老师对我的指导十分重要。两位老师在学术研究与工程实践方面都给予了我很多帮助和指导。两位老师渊博的学识、严谨的治学态度、活跃的学术思想以及温厚的为人,让我受益匪浅。两位老师谆谆的教诲和一丝不苟的指导,是我终身受益的宝贵财富。同时还要感谢实验室杨峰老师、焦连猛老师等众位老师的无私帮助和辛勤栽培。

感谢实验室给予了一个温暖的环境,感谢史志远博士、胡玉梅博士、孙帅硕士、戴安东硕士、张婷婷硕士、祝志勇硕士、马季容硕士、孙藏安硕士、王家琦硕士等实验室的各位师兄、师姐、师弟、师妹,可以帮助我解决各种学术与生活中的问题,与大家的讨论使得我可以更加深入的理解问题,而科研之余的丰富多彩的活动也帮助我劳逸结合。

其次,感谢含辛茹苦的父母多年来对我学业和生活上的关心、支持和理解,没有你们就没有我的今天。感谢李毅兰同学,多年的陪伴使得我成长为一个更加优秀的人。

最后还要感对我的论文进行评审和指导的各位专家老师!