% \section{dfad}
% \centerline{\textbf{\Large{作者硕士期间完成的文章及其他研究成果}}}
\begin{enumerate}
\item 第一作者,发明专利,一种基于深度学习的雷达辐射源类别识别方法,受理号:201711145195.9
\item 第二作者,国防专利,一种天波超视距雷达地海杂波类型识别方法,受理号:201718003477.X

\item 第二作者,国防专利,一种天波超视距雷达多目标自动跟踪方法,受理号:201618010431.6

\item 第二作者,Control Conference (CCC),2016 35th Chinese,Controller design for anti-heeling system in container ships,DOI: 10.1109/ChiCC.2016.7554263

\item 第二作者,International Radar Symposium 2017,Random Sample Consensus Algorithm for Multiple Target Tracking in Over-the-horizon Radar,DOI: 10.23919/IRS.2017.8008099

\item 第二作者,Sea/Land Clutter Recognition for Over-The-Horizon Radar via Deep Convolution Neural Network,Under Review

\item 第三作者,国防专利,一种天波超视距雷达多目标跟踪联合检测与跟踪方法,受理号:201618001893.1

\item 第二作者,软件著作权:LIFT\underline{\hspace{0.5em}}基于变分贝叶斯的多目标联合检测与跟踪软件,登记号:2016R11S233158

\item 第五作者,国防专利,一种基于置信传播的多路径数据关联方法,受理号:201618001088.9

\item 第五作者,IEEE Journal of Selected Topics in Signal Processing,Joint Detection and Tracking for Multipath Targets: A Variational Bayesian Approach,Under Review

\item 主要完成人,天波超视距雷达地海杂波识别与舰船目标跟踪性能提升技术研究,南京14所合作项目

\end{enumerate}

% \centerline{作者硕士期间参与的项目}
