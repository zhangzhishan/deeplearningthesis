% \centerline{\textbf{\Large{作者硕士期间完成的文章及其他研究成果}}}
\textbf{\Large{作者硕士期间完成的文章及其他研究成果}}
\begin{enumerate}
\item 第一作者,发明专利,一种基于深度学习的雷达辐射源类别识别方法,受理号:201711145195.9

\item 第二作者,发明专利,一种天波超视距雷达地海杂波类型识别方法,受理号:201718003477.X

\item 第二作者,发明专利,一种天波超视距雷达多目标自动跟踪方法,受理号:201618010431.6

\item 第二作者,Controller design for anti-heeling system in container ships[C]. In Control Conference (CCC), 2016 35th Chinese. IEEE2016:5798–5803.

\item 第二作者, Random sample consensus algorithm for multiple target tracking in over-the-horizon radar[C]. In Radar Symposium (IRS), 2017 18th International. IEEE2017:1–10.

\item 第二作者,Sea/Land Clutter Recognition for Over-The-Horizon Radar via Deep Convolution Neural Network[J]. \textit{Waiting to Submit}

\item 第三作者,发明专利,一种天波超视距雷达多目标跟踪联合检测与跟踪方法,受理号:201618001893.1

\item 第二作者,软件著作权:LIFT\underline{\hspace{0.5em}}基于变分贝叶斯的多目标联合检测与跟踪软件,登记号:2016R11S233158

\item 第五作者,发明专利,一种基于置信传播的多路径数据关联方法,受理号:201618001088.9

\item 第五作者,Joint Detection and Tracking for Multipath Targets: A Variational Bayesian Approachs[J]. \textit{Submitted to } IEEE Journal of Selected Topics in Signal Processing, 2017, \textit{Under Review}

\end{enumerate}

\textbf{\Large{作者硕士期间参与的项目}}

\begin{enumerate}

\item 地海杂波识别与舰船目标跟踪性能提升技术研究,中电14所合作项目

\item 基于子模优化的远程预警传感器管理研究,青年科学基金项目,基金编号:61503305

\item 高维未知参数下的天波超视距雷达目标跟踪算法研究,国家自然科学基金青年科学项目,基金编号:61501378

\end{enumerate}