\chapter{绪论}
\section{引言}
而是
\section{天波超视距雷达地海杂波识别}
短发
\section{辐射源类别识别}

辐射源的快速、准确和鲁棒的自动目标识别在现代军事中的作用十分重要,自动目标识别算法需要可以准确区分出已知目标和未知目标,同时可以正确的对于已知目标进行分类。我们需要在未收集大量数据的前提下,可以迅速的识别出新的目标。与传统的利用已知类别的样本进行训练测试的机器学习算法不同,我们这个问题是在Open Set的情况下,将需要考虑将输入识别为未知的情况。本文利用深度卷积神经网络与支持向量机进行结合,以雷达信号的模糊函数作为训练样本,构建了一个可以对未知分类进行识别的分类器。我们利用实际数据进行验证,证明我们的分类器具有很强的准确性。
随着科学技术的进步,现代战场形势瞬息万变,信息对抗在现代军事中的作用越来越重要。纵观整个 20 世纪所爆发的两次世界大战和数次局部战争、21 世纪初的美阿、美伊之战以及最近闹得沸沸扬扬的韩国的萨德事件,无一不昭示着现代战争已成为电子战的“天下”,电子战技术也在历次实战演练中逐渐成熟。电子战也称电子对抗,包括电子侦察、电子攻击和电子防护三个方面。电子侦察主要指从敌方雷达及其武器系统获取有用信息,通过雷达辐射源个体识别,可以对战场环境中敌我双方雷达辐射源的分布情况实施侦察,提供更加全面的、精确的电磁斗争与武器的态势,进行有效的战场指挥与决策,雷达辐射源个体识别已成为当前电子战特别是电子侦察领域的研究热点和难点。然而由于辐射源的特征未知、信号波形日趋复杂、战时电磁环境恶劣,给辐射源的精确识别带来了越来越严峻的挑战。

在雷达辐射源信号特征挖掘方面,己有很多学者作了大量研究工作,在上世纪70年代国外相关研究人员就开始了该部分的研究,该部分研究可以分为两个阶段:
第一阶段为辐射源基本参数特征研究。对于原始信号特征直接求取其载波频率、脉冲宽度、脉冲幅度、到达角度和到达时间等信息,利用其中一个或多个作为特征向量。这种情况主要是应用于电磁环境相对单一、辐射源类别较少、信号形式单一、雷达参数固定的早期。

第二阶段自20世纪90年代以来,西方的军事强国开始研究雷达辐射信号的脉内特征,相继提出了了多种分析雷达信号脉内特征的方法。有代表性的工作有:时域波形分析法、谱相关法、基于专家知识信号处理法、时频综合法、小波分析法、信息理论准则与聚类技术综合法、脉内瞬时频率特征与累积法、信号的分型特征等。

国内对雷达辐射源个体识别技术的研究始于上世纪 80 年代初,虽然起步较晚,但受到了军方的高度重视,在“九五”、“十五”和“十一五”国防预研中给予了大力资助。在脉内特征挖掘方面,毕大平提出易于工程实现的脉内瞬时频率提取技术;张葛祥提出了雷达辐射源信号的小波包特征、相像系数特征、熵特征、粗集理论、信息维数和分形盒维数;朱明提出了基于原子分解的特征、基于Chirplet原子的特征、时频原子特征;普运伟提出了瞬时频率派生特征、模糊函数主脊切面特征;陈稻伟提出了符号化脉内特征、围线积分双谱特征等;余志斌提出的局域波分解、小波脊频级联特征。

另一方面,雷达辐射源识别是一个典型的分类问题,其主要思路为在得到辐射源信号的特征表示之后,借助有效的分类算法来实现特征空间到决策空间的转换,从而确定信号的所属类别。大量的分类算法被成功运用于雷达辐射源识别中,如模板匹配、神经网络、支持向量机等。一般被应用于该领域的有三种分类方法,一种是判别型分类器,其需要在学习过程中最优化某种目标函数;另一种为生成模型分类器,其主要是基于先验概率和类别条件概率密度进行估计,如线性判别分类器、K最近邻等;第三种是决策树分类算法,通过人类专家的先验知识进行分类,如ID3、C4.5算法。

\section{论文研究内容及结构}