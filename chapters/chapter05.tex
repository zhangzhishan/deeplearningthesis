\chapter{总结}
\section{本文的主要贡献}
雷达信号处理一直是雷达领域一个很重要的方面,通过对接受到的信号的处理分析,从而获得更多的信息。在电子科技迅猛发展的今天,电磁环境急剧变化,雷达信号处理在军事侦查,目标识别,电子对抗等领域有着广阔的应用场景。而深度学习方法由于其在识别分类等领域的结果,引起了国内外越来越多学者的重视。本文对于雷达信号处理中的两个方面进行了研究,通过对雷达信号数据的分析,构建了一个合理的卷积神经网络模型,实现了对于雷达信号类别的识别。同时,针对于雷达中存在未知类别的情况,构建了一个支持向量机分类器作二次识别,实现了未知分类的辨别。本文的主要研究成果和最终的结论如下:

本文首先综述了深度学习的研究现状和发展,以及其与传统神经网络方法相比的优点所在。描述了深度学习的几种常用方法和基本结构,研究了深度卷积神经网络的原理和其训练过程。

针对于天波超视距雷达中由于电离层的变化无法确定其PD变换系数的问题,本文通过地海杂波识别进行地理位置匹配来获取该变换系数,用于提高目标定位精度。在地海杂波识别过程中,通过对于海量频谱数据的分析,构建了一个一维卷积神经网络分类器。同时将该算法的识别结果与传统的基于信号频谱分析和支持向量机算法进行对比,证明了本文算法的优越性。

对于辐射源识别中未知分类的辨别问题,首先概述了辐射源识别中雷达信号处理的过程,介绍了辐射源特征提取和辐射源识别的常用方法及存在的问题。本文将深度学习和支持向量机相结合,构建了用于辐射源分类及未知分类辨别的模型,通过实际数据进行实验验证,并对不同辐射源类别数、卷积神经网络层数、节点数、不同支持向量机参数与正确率进行了比较,讨论了相关参数对结果的影响。 

\section{后续的研究进展}
在本文中,我们已经利用实际数据进行测试并验证了我们的深度学习方法在地海杂波识别方法和辐射源识别这两个方面的有效性。但是由于雷达数据处理方面信号种类多,在实际工程中涉及环节多,因此,对于利用深度学习进行雷达信号处理还有很多工作需要完成。结合本文所研究的问题,我们希望未来可以在以下方面做进一步研究:
\begin{itemize}
	\item 针对于雷达信号特征,对本文算法进一步调整。在天波雷达地海杂波识别中,我们现在把已有数据分成几组,分别进行训练。我们计划尝试使用一些方法对这些数据进行融合分析,以获得一个可以适应各种数据情况的模型。对于辐射源识别,利用更多的数据对算法进行进一步的验证,目前类别较少的情况下,部分结果会对选择的类别具有一定的依赖性,另一方面是选取更多更合适的特征进行训练学习。
	\item 进一步优化算法,提高计算效率。虽然在数据量比较大的情况下,深度学习算法具有准确率高的优势,但训练过程计算量较大。本文拟将深度学习方法同其他方法进行结合,进一步完善网络结构,较少计算量、提高训练速度。 
	\item 进一步优化网络结构、相关参数的选取、训练方法等。深度学习的理论研究仍然存在一些不足,可以通过进一步的理论研究,选择更加优秀的参数和训练方法。除了本文的卷积神经网络结构之外,我们计划尝试其他深度学习的方法或者思想来解决我们的问题或者构建更优的神经网络结构。
	\item 对数据进行无监督或者半监督学习的方式进行训练。由于雷达信号量大,人为进行标记困难较大,本文计划进一步尝试无监督等减少人为标记的工作量。
\end{itemize}
