\chapter{总结}
\section{本文的主要贡献}

\section{后续的研究进展}
<<<<<<< HEAD
我们希望未来在目前的基础上,在下面几方面做进一步的研究:(1)利用更多的数据对算法进行进一步的验证,目前类别较少的情况下,部分结果会对选择的类别具有一定的依赖性,另一方面是选取更多更合适的特征进行训练学习;(2)利用集成学习提高学习精度;(3)尝试新方法对于未知分类识别;(4)对于未知分类,尝试利用无监督或者半监督学习进行训练学习。
=======

% In this paper, we have tested and verified the effectiveness of our convolution neural network sea/land recolonization method. There are still some further studies of the workplace:
在本文中,我们已经测试和验证了我们的卷积神经网络海陆重建方法的有效性。 还有一些关于工作场所的研究:
\begin{itemize}
	% \item In addition to our convolution neural network structure, we can try other neural networks(i.e. RNN, which is a usual method for time series data) or build a CNN with other structures.
	% \item We now divided our data into several groups and train them separately. We can try to use some methods to fusion these data and generate a model that can fit in every problem.
	% \item We also intend to perform unsupervised algorithms to unlabeled data to pre-train the networks.
	\item 除了我们的卷积神经网络结构之外,我们可以尝试其他神经网络(即RNN,这是时间序列数据的常用方法),也可以用其他结构构建CNN。
	\item 我们现在把我们的数据分成几组,分别进行培训。 我们可以尝试使用一些方法来融合这些数据,并生成一个可以适应每个问题的模型。
	\item 我们还打算对未标记的数据执行无监督的算法来预先训练网络。
\end{itemize}
>>>>>>> 5fe069c16b36ca12ac6e28cad522adc41c98413b
