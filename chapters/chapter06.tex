% !Mode:: "TeX:UTF-8"

\chapter{总结}
\label{sec:paper_summary}
\section{本文的主要贡献}
雷达信号分类一直是雷达领域一个很重要的方面,通过对接受到的信号的分类识别,可以更好的了解战场形势。在电子科技迅猛发展的今天,电磁环境急剧变化,雷达信号分类在军事侦查、目标识别、电子对抗等领域有着广阔的应用前景。
而深度学习方法由于其在识别分类等领域的优势,引起了国内外越来越多学者的重视。本文对雷达信号分类中的天波雷达地海杂波识别和辐射源识别这两个方面进行了研究,通过对雷达信号数据的分析,构建合理的卷积神经网络模型,实现了雷达信号类别的识别,并尝试使用无监督方法进行聚类识别。另一方面,针对辐射源识别过程中存在未知类别的情况,构建了一个支持向量机分类器作为Meta-Recognition对深度卷积神经网络的识别结果进行二次处理,实现了未知分类的辨别。
本文的主要研究成果和最终的结论如下:

本文首先叙述了深度学习的研究现状和发展及与传统神经网络方法相比的优点所在,同时描述了深度学习的几种常用方法和基本结构,研究了深度卷积神经网络的原理和其训练过程。

针对天波雷达中由于电离层的变化无法确定其坐标配准系数的问题,本文通过地海杂波识别结果进行地理位置匹配来获取该变换系数,用于提高目标定位精度。在地海杂波识别过程中,通过对海量频谱数据的分析,构建了一个一维卷积神经网络分类器。将该算法的识别结果与传统的基于信号频谱分析和支持向量机算法的结果进行对比,证明了本文算法的优越性。

针对天波雷达地海杂波识别过程中人工标定复杂且无法准确标定的情况,本文提出了一种深度嵌入卷积聚类方法。利用卷积自编码器提取深度特征,然后通过K-均值算法获得初始聚类中心,通过本文设计的损失函数使用小批量随机梯度下降算法优化聚类中心。

针对辐射源识别中未知分类的辨别问题,首先概述了辐射源识别中雷达信号处理的过程,介绍了辐射源特征提取和辐射源识别的常用方法及存在的问题。本文将深度学习和支持向量机结合,构建了用于辐射源分类及未知分类辨别的模型,通过实际数据进行实验验证,并针对不同的参数对结果进行了分析。
\section{后续的研究进展}
本文测试验证了深度学习方法在地海杂波识别和辐射源识别这两个方面的有效性。但是由于雷达数据处理方面信号种类多,在实际工程中涉及环节多,因此,利用深度学习进行雷达信号分类还有很多工作需要完成。
结合本文所研究的问题,希望未来可以在以下方面做进一步研究:
\begin{itemize}
\item 进一步优化算法,提高计算效率。虽然在数据量比较大的情况下,深度学习算法具有准确率高的优势,但训练过程计算量较大。本文拟进一步优化网络结构、相关参数的选取、训练方法或者尝试其他深度学习的方法或者思想来解决本文的问题,以降低计算量、提高训练速度。
\item 结合地海杂波识别的问题特征,对深度卷积神经网络进行调整。地海杂波识别目前仍有两个难以解决的问题,一个是对地海混合区域的识别,由于天波雷达的分辨率不高,会存在一些分辨单元中既包括海又包括地,这些单元在匹配阶段重要程度比较高,需要进一步探索对这种单元准确识别的方法;另一个是如何同时识别不同分辨率的数据,目前将不同分辨率的数据划分成不同的数据集分别训练与识别,需要进一步寻找一种网络结构可以同时处理这些数据。
\item 利用集成学习进一步提高识别精度。本文均只设计了单独的模型解决问题,虽然已经取得了较高的准确率,但是可以利用集成学习的思想,设计多个分类器,以某种规则整合所有结果,从而获得更好的结果。
\item 对深度学习的稳定性进行分析。由于在工程应用中,需要明确算法的适用范围,因此需要对本文提出的深度学习算法的边界条件进行分析,可以准确判别出由于自然环境或其余因素导致的已训练的模型不再适用的情况,从而可以采取其余措施进行修正。
\item 充分利用先验信息提高识别精度。在地海杂波识别中会存在连续地理分辨单元上杂波特性的相关性,而辐射源识别中也会存在部分辐射源的信息已知,如民航飞机GPS信息的ADS-B数据,下一步打算将各种先验信息与本文提出的算法进行融合,提高识别精度。

\end{itemize}
