\chapter{总结}
\label{sec:paper_summary}
\section{本文的主要贡献}
雷达信号分类一直是雷达领域一个很重要的方面,通过对接受到的信号的分类识别,可以更好的了解战场形势。在电子科技迅猛发展的今天,电磁环境急剧变化,雷达信号分类在军事侦查、目标识别、电子对抗等领域有着广阔的应用前景。
而深度学习方法由于其在识别分类等领域的优势,引起了国内外越来越多学者的重视。本文对雷达信号分类中的天波雷达地海杂波识别和辐射源识别这两个方面进行了研究,通过对雷达信号数据的分析,构建合理的卷积神经网络模型,实现了雷达信号类别的识别,并尝试利用无监督方法进行聚类识别。同时,针对识别过程中存在未知类别的情况,构建了一个支持向量机分类器作为Meta-Recognition对深度卷积神经网络的识别结果进行二次处理,实现了未知分类的辨别。
本文的主要研究成果和最终的结论如下:

本文首先叙述了深度学习的研究现状和发展,以及其与传统神经网络方法相比的优点所在。描述了深度学习的几种常用方法和基本结构,研究了深度卷积神经网络的原理和其训练过程。

针对天波超视距雷达中由于电离层的变化无法确定其坐标配准系数的问题,本文通过地海杂波识别进行地理位置匹配来获取该变换系数,用于提高目标定位精度。在地海杂波识别过程中,通过对海量频谱数据的分析,构建了一个一维卷积神经网络分类器。同时将该算法的识别结果与传统的基于信号频谱分析和支持向量机算法的结果进行对比,证明了本文算法的优越性。

针对天波超视距雷达地海杂波识别过程中人工标定复杂且无法准确标定的情况,本文提出了一种深度嵌入卷积聚类方法。利用卷积自编码器进行深度特征提取,然后利用K-均值算法获得初始聚类中心,通过本文设计的损失函数利用小批量随机梯度下降算法对聚类中心进行优化。同时利用实际数据对本文提出的聚类算法进行了验证。

针对辐射源识别中未知分类的辨别问题,首先概述了辐射源识别中雷达信号处理的过程,介绍了辐射源特征提取和辐射源识别的常用方法及存在的问题。本文将深度学习和支持向量机结合,构建了用于辐射源分类及未知分类辨别的模型,通过实际数据进行实验验证,并针对不同的参数对结果进行了分析。
\section{后续的研究进展}
本文利用实际数据进行测试验证了深度学习方法在地海杂波识别和辐射源识别这两个方面的有效性。但是由于雷达数据处理方面信号种类多,在实际工程中涉及环节多,因此,利用深度学习进行雷达信号分类还有很多工作需要完成。结合本文所研究的问题,希望未来可以在以下方面做进一步研究:
\begin{itemize}
	\item 进一步优化算法,提高计算效率。虽然在数据量比较大的情况下,深度学习算法具有准确率高的优势,但训练过程计算量较大。本文拟进一步优化网络结构、相关参数的选取、训练方法或者尝试其他深度学习的方法或者思想来解决本文的问题,以降低计算量、提高训练速度。
	\item 进一步加强对深度学习理论知识的研究。本文目前的工作更偏重于工程应用,对深度学习相关理论知识的研究不足,后续需要增强理论知识的研究,实现在理论层面的创新。
\end{itemize}
