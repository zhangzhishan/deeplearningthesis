\documentclass{standalone}
    \usepackage{xeCJK}	

\usepackage{tikz}
\setCJKmainfont{STFangsong}	
\usetikzlibrary{matrix,chains,positioning,decorations.pathreplacing,arrows}

\begin{document}
\pagestyle{empty}

\def\layersep{2.5cm}

\begin{tikzpicture}[shorten >=1pt,->,draw=black!50, node distance=\layersep]
    \tikzstyle{every pin edge}=[<-,shorten <=1pt]
    \tikzstyle{neuron}=[circle,fill=black!25,minimum size=17pt,inner sep=0pt]
    \tikzstyle{input neuron}=[neuron, fill=green!50];
    \tikzstyle{output neuron}=[neuron, fill=red!50];
    \tikzstyle{hidden neuron}=[neuron, fill=blue!50];
    \tikzstyle{annot} = [text width=8em, text centered]

    % Draw the input layer nodes
    \foreach \name / \y in {1,...,4}
    % This is the same as writing \foreach \name / \y in {1/1,2/2,3/3,4/4}
        \node[input neuron, pin=left:输入 $x_\y$] (I-\name) at (0,-\y) {};

    % Draw the hidden layer nodes
    \foreach \name / \y in {1,...,5}
        \path[yshift=0.5cm]
            node[hidden neuron] (H1-\name) at (\layersep,-\y cm) {};

    % Draw the hidden layer nodes
    \foreach \name / \y in {1,...,5}
    \path[yshift=0.5cm]
        node[hidden neuron] (H2-\name) at (2 * \layersep,-\y cm) {};

    % Draw the hidden layer nodes
    \foreach \name / \y in {1,...,5}
    \path[yshift=0.5cm]
        node[hidden neuron] (H3-\name) at (3 * \layersep,-\y cm) {};

    % Draw the output layer node
    \foreach \name / \y in {1/2,2/3,3/4}
        \node[output neuron, pin={[pin edge={->}]right:输出 $y_\name$}, right of=H3-\y] (O-\name) {};

    % Connect every node in the input layer with every node in the
    % hidden layer.
    \foreach \source in {1,...,4}
        \foreach \dest in {1,...,5}
            \path (I-\source) edge (H1-\dest);

    \foreach \source in {1,...,5}
        \foreach \dest in {1,...,5}
            \path (H1-\source) edge (H2-\dest);
        
    \foreach \source in {1,...,5}
        \foreach \dest in {1,...,5}
            \path (H2-\source) edge (H3-\dest);
    % Connect every node in the hidden layer with the output layer
    \foreach \source in {1,...,5}
        \foreach \dest in {1,2,3}
            \path (H3-\source) edge (O-\dest);

    % Annotate the layers
    \node[annot,above of=H1-1, node distance=1cm] (hl) {隐层 $L_2$};
    \node[annot,above of=H2-1, node distance=1cm] (h2) {隐层 $L_3$};
    \node[annot,above of=H3-1, node distance=1cm] (h3) {隐层 $L_4$};
    
    \node[annot,left of=hl] {输入层 $L_1$};
    \node[annot,right of=h3] {输出层 $L_5$};
\end{tikzpicture}
% End of code
\end{document}